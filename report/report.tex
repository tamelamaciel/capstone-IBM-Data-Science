%
% LaTeX report template 
%
\documentclass[a4paper,10pt]{article}
\usepackage{graphicx}
\usepackage[english]{babel}
\usepackage[latin1]{inputenc}
\usepackage{hyperref}
%
\begin{document}
%
   \title{Where to open an Indian restaurant in Ireland?}

   \author{Tamela Maciel}
          
   \date{}

   \maketitle
   
   %\tableofcontents
 
  %\newpage

\section{Introduction and Business Problem}
The introduction should state clearly why the study was started
and give a relatively short and essential overview of the topic
you are exploring. References to previous works can be made here.
 
The introduction should not contain the conclusions. 
At the end of the introduction the outline of the paper may be described.
 
 
\section{Data}
Our client runs several successful Indian restaurants in Leicester, UK, and seeks to reproduce the community and success in a new city or town in Ireland. In order to match the criteria of the client, we identify the best new location based on 1) the relative size of the city, 2) the existing profile of restaurants (Indian or other) in that city, and 3) the proportion of residents who are Asian or Irish Asian. The data for these criteria comes from a mixture of census data and Foursquare venue location data.

We use population statistics from the 2016 Irish census, scraped from the Wikipedia table: List of urban areas in the Republic of Ireland by population\footnote{\url{https://en.wikipedia.org/wiki/List_of_urban_areas_in_the_Republic_of_Ireland_by_population}}, and the 2011 UK census, from the Wikipedia table: List of urban areas in the United Kingdom\footnote{\url{https://en.wikipedia.org/wiki/List_of_urban_areas_in_the_United_Kingdom}}, in order to profile all Irish urban centres that are the same relative size of Leicester or bigger.

In order to capture the restaurant profile of a city, we need a large, up-to-date database of location-based venue data. In this analysis, we use the Foursquare API to return all restaurants within XXX metres of geographical centre of the top X Irish cities, as well as the city of Leicester, UK for comparison.

For a new restaurant to be a success, it should locate itself in a place where there is a market gap, but also an existing base of future customers. We assume that areas with higher populations of south Asian immigration in Ireland will be more receptive to the authentic Indian cuisine that our client prides himself on. 

Counties are ranked by the relative sizes of their Asian communities using the 2016 Irish census. Ethnicity by county is recorded on the Central Statistics Office webpage in table E8001: Irish Travellers Ethnicity and Religion\footnote{\url{https://statbank.cso.ie/px/pxeirestat/Statire/SelectVarVal/Define.asp?maintable=E8001&PLanguage=0}}. We exclude the `Asian or Asian Irish - Chinese' category, and select by `Asian or Asian Irish - any other Asian background'. Note that the census tables only summarises ethnicity to the county level, and not for individual cities or towns. However, given the counties are relatively small, this is suitable for the purposes of making a recommendation on new restaurant location.




\section{Model?}

Foursquare API - 'explore' returns more of the results expected (i.e. when 'search' was used for Leicester, Kayal wasn't on the list).
Note a limit of 100 per search query - given depth of dublin restaurants, this means that not many indian places turn up.
I could use the 'offset' function to step through the results.

Otherwise, to make it more fair, I use 'explore' and return top 100 results sorted by popularity. so we're getting a truer picture of whether Indian restaurants are popular in the town.


\section{Results}
In this section you present your findings and results.


\newpage
\section{Tables and figures}
Figures demonstrate and prove conclusions. They should convince 
the reader, preferably at first glance. Figures should be self-explanatory. 
The legends should have a well-defined meaning. The lettering and the 
thickness of lines and symbols should be large enough to remain recognizable 
after printing.

The figure captions should contain all the information needed to 
understand the data presented and references to the text of the paper 
should be minimized.


%\begin{figure}[htb]
%   \centering
%   \includegraphics[width=8cm]{example.ps}
%      \caption{Tables and figures are floating objects, \LaTeX\, will place them
%      where it thinks it's best (whatever that means\ldots)
%              }
%         \label{FigVibStab}
%\end{figure}


\section{Conclusion}
Here you summarize the essential aspects and findings 
of your work and analysis.


\end{document}

